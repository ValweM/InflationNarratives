% !TeX spellcheck = en_US
\section{Conclusion}\label{sec:Conclusion}


This paper proposes a new methodological approach to measure known inflation narratives in news reports. The combination of survey information on inflation narratives with a supervised topic model and a latent semantic scaling approach provides information on the prevalence and spread of narratives in a large text corpus, including the Wall Street Journal. For the recent inflationary period, our descriptive results highlight the presence of narratives about changing demand, supply factors including the supply chain, energy prices, and labor shortages, as well as stories about the war in Ukraine and corporate profits. Conversely, narratives about monetary policy, government debt, and the pandemic were prevalent in the preceding period of low inflation and deflation, respectively. To further investigate the relevance for macroeconomic development, additional time series analyses were conducted. As suggested by the multivariate Granger causality tests, the considered narratives contain relevant information for households' expectations. This is in line with the theoretical arguments presented in \cite{Shiller.2017, Tuckett.2020} and \cite{Beckert.2016}. While our estimates suggest only small differences between short- and medium-term aggregate expectations, we highlight heterogeneity across socioeconomic groups. We document notable differences in income, education, and age, especially for medium-term expectations. For example, our results imply that the energy and corporate profits narrative is the main driver of 3-year expectations for households with lower annual incomes, while several narratives Granger-cause the expectations of high-income households. In addition, our analysis points to the importance of age as a driver of heterogeneity. To further investigate potential differences in the pathways of responses to narrative diffusion, we conduct impulse responses. Our estimates show that the responses to 1-year expectations are more pronounced and often more persistent. Moreover, when comparing shocks across narratives, we notice more anchored expectations with respect to a shock in the supply chain, demand shift, and profits narratives.

In summary, our paper reveals the powerful role  that media narratives play in shaping economic expectations, potentially anchoring or unanchoring inflation expectations over time. This impact varies by narrative type and socioeconomic backgrounds of households, making it  particularly relevant for monetary policymakers, who may pay even more attention to media coverage. By recognizing the diverse impact of these narratives, policymakers can develop more targeted communication strategies. Future advances in narrative analysis hold the promise of enabling even more responsive and adaptable policy interventions, aligned with the evolving narratives captured in media.